\subsection*{About}

(c) 2013-\/2020 Lesley De Cruz and Jonathan Demaeyer

See \href{../../LICENSE.txt}{\tt L\+I\+C\+E\+N\+S\+E.\+txt} for license information.

This software is provided as supplementary material with\+:


\begin{DoxyItemize}
\item De Cruz, L., Demaeyer, J. and Vannitsem, S.\+: The Modular Arbitrary-\/\+Order Ocean-\/\+Atmosphere Model\+: M\+A\+O\+O\+AM v1.\+0, Geosci. Model Dev., 9, 2793-\/2808, \href{http://dx.doi.org/10.5194/gmd-9-2793-2016}{\tt doi\+:10.\+5194/gmd-\/9-\/2793-\/2016}, 2016.
\end{DoxyItemize}

{\bfseries Please cite this article if you use (a part of) this software for a publication.}

The authors would appreciate it if you could also send a reprint of your paper to \href{mailto:lesley.decruz@meteo.be}{\tt lesley.\+decruz@meteo.\+be}, \href{mailto:jonathan.demaeyer@meteo.be}{\tt jonathan.\+demaeyer@meteo.\+be} and \href{mailto:svn@meteo.be}{\tt svn@meteo.\+be}.

Consult the M\+A\+O\+O\+AM \href{http://www.github.com/Climdyn/MAOOAM}{\tt code repository} for updates, and \href{http://climdyn.meteo.be}{\tt our website} for additional resources.

A pdf version of this manual is available \href{../latex/Reference_manual.pdf}{\tt here}.





\subsection*{Installation}

The program can be installed with Makefile. We provide configuration files for two compilers \+: gfortran and ifort.

By default, gfortran is selected. To select one or the other, simply modify the Makefile accordingly or pass the C\+O\+M\+P\+I\+L\+ER flag to {\ttfamily make}.

To install, unpack the archive in a folder or clone with git\+:


\begin{DoxyCode}
git clone https://github.com/Climdyn/MAOOAM.git
cd MAOOAM/fortran
\end{DoxyCode}


and run\+:


\begin{DoxyCode}
make
\end{DoxyCode}


The command


\begin{DoxyCode}
make clean
\end{DoxyCode}
 removes the compiled files.

For Windows users, a minimalistic G\+NU development environment (including gfortran and make) is available at \href{http://www.mingw.org}{\tt www.\+mingw.\+org} .





\subsection*{Description of the files}

The model tendencies are represented through a tensor class called Atm\+Oc\+Tensor (\hyperlink{structaotensor__def_1_1atmoctensor}{aotensor\+\_\+def\+::atmoctensor}) which includes all the coefficients. In the standard implementation using \hyperlink{maooam_8f90_source}{maooam.\+f90} , this tensor is computed once at the program initialization.


\begin{DoxyItemize}
\item \hyperlink{maooam_8f90_source}{maooam.\+f90} \+: Main program.
\item \hyperlink{model__def_8f90_source}{model\+\_\+def.\+f90} \+: Main model class module.
\item \hyperlink{aotensor__def_8f90_source}{aotensor\+\_\+def.\+f90} \+: Tensor class Atm\+Oc\+Tensor module.
\item \hyperlink{inprod__analytic_8f90_source}{inprod\+\_\+analytic.\+f90} \+: Inner products class module.
\item \hyperlink{integrator__def_8f90_source}{integrator\+\_\+def.\+f90} \+: A module holding the model\textquotesingle{}s integrator base class definition.
\item \hyperlink{rk2__integrator_8f90_source}{rk2\+\_\+integrator.\+f90} \+: A module which contains the Heun integrator class for the model equations.
\item \hyperlink{rk2__tl__integrator_8f90_source}{rk2\+\_\+tl\+\_\+integrator.\+f90} \+: Heun Tangent Linear (TL) model integrator class module.
\item \hyperlink{rk2__ad__integrator_8f90_source}{rk2\+\_\+ad\+\_\+integrator.\+f90} \+: Heun Adjoint (AD) model integrator class module.
\item \hyperlink{rk4__integrator_8f90_source}{rk4\+\_\+integrator.\+f90} \+: A module which contains the R\+K4 integrator class for the model equations.
\item \hyperlink{rk4__tl__integrator_8f90_source}{rk4\+\_\+tl\+\_\+integrator.\+f90} \+: R\+K4 Tangent Linear (TL) model integrators module.
\item \hyperlink{rk4__ad__integrator_8f90_source}{rk4\+\_\+ad\+\_\+integrator.\+f90} \+: Adjoint (AD) model integrators module.
\item Makefile \+: The Makefile.
\item \hyperlink{params_8f90_source}{params.\+f90} \+: The model parameters classes module.
\item \hyperlink{tl__ad__tensor_8f90_source}{tl\+\_\+ad\+\_\+tensor.\+f90} \+: Tangent Linear (TL) and Adjoint (AD) model tensors class definition module.
\item \hyperlink{test__tl__ad_8f90_source}{test\+\_\+tl\+\_\+ad.\+f90} \+: Tests for the Tangent Linear (TL) and Adjoint (AD) model versions.
\item R\+E\+A\+D\+M\+E.\+md \+: A read me file.
\item L\+I\+C\+E\+N\+S\+E.\+txt \+: The license text of the program.
\item \hyperlink{util_8f90_source}{util.\+f90} \+: A module with various useful functions.
\item \hyperlink{tensor__def_8f90_source}{tensor\+\_\+def.\+f90} \+: Main tensor class utility module.
\item \hyperlink{stat_8f90_source}{stat.\+f90} \+: A module implementing a statistics accumulator class.
\item params.\+nml \+: A namelist to specify the model parameters.
\item int\+\_\+params.\+nml \+: A namelist to specify the integration parameters.
\item modeselection.\+nml \+: A namelist to specify which spectral decomposition will be used.
\end{DoxyItemize}





\subsection*{Usage}

The user first has to fill the params.\+nml and int\+\_\+params.\+nml namelist files according to their needs. Indeed, model and integration parameters can be specified respectively in the params.\+nml and int\+\_\+params.\+nml namelist files. Some examples related to already published article are available in the \href{../../params/}{\tt params} folder.

The modeselection.\+nml namelist can then be filled \+:
\begin{DoxyItemize}
\item N\+B\+OC and N\+B\+A\+TM specify the number of blocks that will be used in respectively the ocean and the atmosphere. Each block corresponds to a given x and y wavenumber.
\item The O\+MS and A\+MS arrays are integer arrays which specify which wavenumbers of the spectral decomposition will be used in respectively the ocean and the atmosphere. Their shapes are O\+M\+S(\+N\+B\+O\+C,2) and A\+M\+S(\+N\+B\+A\+T\+M,2).
\item The first dimension specifies the number attributed by the user to the block and the second dimension specifies the x and the y wavenumbers.
\item The V\+D\+DG model is given as a default example. It is described in\+:
\begin{DoxyItemize}
\item Vannitsem, S., Demaeyer, J., De Cruz, L., and Ghil, M.\+: Low-\/frequency variability and heat transport in a loworder nonlinear coupled ocean-\/atmosphere model, Physica D\+: Nonlinear Phenomena, 309, 71-\/85, \href{https://doi.org/10.1016/j.physd.2015.07.006}{\tt doi\+:10.\+1016/j.physd.\+2015.\+07.\+006}, 2015.
\end{DoxyItemize}
\item Note that the variables of the model are numbered according to the chosen order of the blocks.
\end{DoxyItemize}

Finally, the I\+C.\+nml file specifying the initial condition should be defined. To obtain an example of this configuration file corresponding to the model you have previously defined, simply delete the current I\+C.\+nml file (if it exists) and run the program \+: \begin{DoxyVerb}./maooam
\end{DoxyVerb}


It will generate a new one and start with the 0 initial condition. If you want another initial condition, stop the program, fill the newly generated file and restart \+: \begin{DoxyVerb}./maooam
\end{DoxyVerb}


It will generate two files \+:
\begin{DoxyItemize}
\item evol\+\_\+field.\+dat \+: the recorded time evolution of the variables.
\item mean\+\_\+field.\+dat \+: the mean field (the climatology)
\end{DoxyItemize}

By default, the code uses the \hyperlink{namespacerk2__integrator}{rk2\+\_\+integrator} class of integrator, which integrates the model with the \href{https://en.wikipedia.org/wiki/Heun%27s_method}{\tt Heun algorithm}. However, by modifying the file \hyperlink{maooam_8f90_source}{maooam.\+f90}, it is possible to use the \hyperlink{namespacerk4__integrator}{rk4\+\_\+integrator} class which integrates the model with the \href{https://en.wikipedia.org/wiki/Runge%E2%80%93Kutta_methods}{\tt fourth-\/order Runge-\/\+Kutta algorithm (R\+K4)}. It is also possible to write an user-\/defined integrator by subclassing the base class \hyperlink{structintegrator__def_1_1integrator}{integrator\+\_\+def\+::integrator}.

The tangent linear and adjoint models of M\+A\+O\+O\+AM are provided in the \hyperlink{namespacetl__ad__tensor}{tl\+\_\+ad\+\_\+tensor}, with integrators provided in the \hyperlink{namespacerk2__tl__integrator}{rk2\+\_\+tl\+\_\+integrator}, \hyperlink{namespacerk2__ad__integrator}{rk2\+\_\+ad\+\_\+integrator}, \hyperlink{namespacerk4__tl__integrator}{rk4\+\_\+tl\+\_\+integrator} and \hyperlink{namespacerk4__ad__integrator}{rk4\+\_\+ad\+\_\+integrator} modules. It is documented \href{./md_doc_tl_ad_doc.html}{\tt here}.





\subsection*{Implementation notes}

As the system of differential equations is at most bilinear in $y_j$ ( $j=1..n$), $\boldsymbol{y}$ being the array of variables, it can be expressed as a tensor contraction \+:

\[ \frac{d y_i}{dt} = \sum_{j,k=0}^{ndim} \, \mathcal{T}_{i,j,k} \, y_k \; y_j \]

with $y_0 = 1$.

The tensor aotensor\+\_\+def\+::aotensor is the tensor $\mathcal{T}$ that encodes the differential equations is composed so that\+:


\begin{DoxyItemize}
\item $\mathcal{T}_{i,j,k}$ contains the contribution of $dy_i/dt$ proportional to $ y_j \, y_k$.
\item Furthermore, $y_0$ is always equal to 1, so that $\mathcal{T}_{i,0,0}$ is the constant contribution to $dy_i/dt$
\item $\mathcal{T}_{i,j,0} + \mathcal{T}_{i,0,j}$ is the contribution to $dy_i/dt$ which is linear in $y_j$.
\end{DoxyItemize}

The tensor \hyperlink{structaotensor__def_1_1atmoctensor}{aotensor\+\_\+def\+::atmoctensor} is composed as an upper triangular matrix (in the last two coordinates), and its computation uses the inner products defined in a \hyperlink{structinprod__analytic_1_1innerproducts}{inprod\+\_\+analytic\+::innerproducts} class.

The implementation is made using Fortran classes that are linked together. It turns the model into an instanciated object that can be reused, allowing the usage of several different model versions in the same program. See the page \hyperlink{structmodel__def_1_1model}{model\+\_\+def\+::model} for a sketch of how the various classes are linked together.





\subsection*{Final Remarks}

The authors would like to thank Kris for help with the lua2fortran project. It has greatly reduced the amount of (error-\/prone) work.

No animals were harmed during the coding process. 