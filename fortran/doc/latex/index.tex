\subsection*{About}

(c) 2013-\/2016 Lesley De Cruz and Jonathan Demaeyer

See \hyperlink{LICENSE_8txt}{L\-I\-C\-E\-N\-S\-E.\-txt} for license information.

This software is provided as supplementary material with\-:


\begin{DoxyItemize}
\item De Cruz, L., Demaeyer, J. and Vannitsem, S.\-: A modular arbitrary-\/order ocean-\/atmosphere model\-: M\-A\-O\-O\-A\-M, Geosci. Model Dev. Discuss., \href{http://dx.doi.org/xx/xxx}{\tt doi\-:xx/xxx}, 2016.
\end{DoxyItemize}

{\bfseries Please cite this article if you use (a part of) this software for a publication.}

The authors would appreciate it if you could also send a reprint of your paper to \href{mailto:lesley.decruz@meteo.be}{\tt lesley.\-decruz@meteo.\-be}, \href{mailto:jonathan.demaeyer@meteo.be}{\tt jonathan.\-demaeyer@meteo.\-be} and \href{mailto:svn@meteo.be}{\tt svn@meteo.\-be}.

Consult the M\-A\-O\-O\-A\-M code repository at \href{http://dx.doi.org/yy/yyy}{\tt doi\-:yy/yyy} for updates, and \href{http://climdyn.meteo.be}{\tt our website} for additional resources.



\subsection*{Installation}

The code should be compiled with gfortran 4.\-7+ (allows for allocatable arrays in namelists). Unpack the archive in a folder, and run\-: make

Remark\-: The command \char`\"{}make clean\char`\"{} removes the compiled files.

For Windows users, a minimalistic G\-N\-U development environment (including gfortran and make) is available at \href{http://www.mingw.org}{\tt www.\-mingw.\-org} . 



\subsection*{Description of the files}

The model tendencies are represented through a tensor called aotensor which includes all the coefficients. This tensor is computed once at the program initialization.


\begin{DoxyItemize}
\item \hyperlink{maooam_8f90}{maooam.\-f90} \-: Main program.
\item \hyperlink{aotensor__def_8f90}{aotensor\-\_\-def.\-f90} \-: Tensor aotensor computation module.
\item I\-C\-\_\-def.\-f90 \-: A module which loads the user specified initial condition.
\item \hyperlink{inprod__analytic_8f90}{inprod\-\_\-analytic.\-f90} \-: Inner products computation module.
\item \hyperlink{integrator_8f90}{integrator.\-f90} \-: A module which contains the integrator for the model equations.
\item Makefile \-: The Makefile.
\item \hyperlink{params_8f90}{params.\-f90} \-: The model parameters module.
\item \hyperlink{maooam__tl__ad_8f90}{maooam\-\_\-tl\-\_\-ad.\-f90} \-: Tangent Linear (T\-L) and Adjoint (A\-D) model tensors definition module
\item \hyperlink{tl__ad__integrator_8f90}{tl\-\_\-ad\-\_\-integrator.\-f90} \-: Tangent Linear (T\-L) and Adjoint (A\-D) model integrators module
\item \hyperlink{test__tl__ad_8f90}{test\-\_\-tl\-\_\-ad.\-f90} \-: Tests for the Tangent Linear (T\-L) and Adjoint (A\-D) model versions
\item R\-E\-A\-D\-M\-E.\-md \-: A read me file.
\item \hyperlink{LICENSE_8txt}{L\-I\-C\-E\-N\-S\-E.\-txt} \-: The license text of the program.
\item \hyperlink{util_8f90}{util.\-f90} \-: A module with various useful functions.
\item \hyperlink{stat_8f90}{stat.\-f90} \-: A module for statistic accumulation.
\item params.\-nml \-: A namelist to specify the model parameters.
\item int\-\_\-params.\-nml \-: A namelist to specify the integration parameters.
\item modeselection.\-nml \-: A namelist to specify which spectral decomposition will be used. 


\end{DoxyItemize}

\subsection*{Usage}

The user first has to fill the params.\-nml and int\-\_\-params.\-nml namelist files according to their needs.

The modeselection.\-nml namelist can then be filled \-:
\begin{DoxyItemize}
\item \hyperlink{classparams_a54123b5a947703d21d0c882dec6780ac}{params\-::nboc} and \hyperlink{classparams_aa5dc201b0a59d8bb25a5dc52d2ed3cac}{params\-::nbatm} specify the number of blocks that will be used in respectively the ocean and the atmosphere. Each block corresponds to a given x and y wavenumber.
\item The \hyperlink{classparams_a5e25e072992d5908eea5308243b7ec63}{params\-::oms} and \hyperlink{classparams_aa95299f1a9c54693b85e049004369089}{params\-::ams} arrays are integer arrays which specify which wavenumbers of the spectral decomposition will be used in respectively the ocean and the atmosphere. Their shapes are oms(nboc,2) and ams(nbatm,2).
\item The first dimension specifies the number attributed by the user to the block and the second dimension specifies the x and the y wavenumbers.
\item The V\-D\-D\-G model, described in Vannitsem et al. (2015) is given as an example in the archive.
\item Note that the variables of the model are numbered according to the chosen order of the blocks.
\end{DoxyItemize}

Model and integration parameters can be specified in the params.\-nml namelist file.

Finally, the I\-C.\-nml file specifying the initial condition should be defined. To obtain an example of this configuration file corresponding to the model you have previously defined, simply delete the current I\-C.\-nml file (if it exists) and run the program \-: \begin{DoxyVerb}./maooam
\end{DoxyVerb}


It will generate a new one and start with the 0 initial condition. If you want another initial condition, stop the program, fill the newly generated file and restart \-: \begin{DoxyVerb}./maooam
\end{DoxyVerb}


It will generate two files \-:
\begin{DoxyItemize}
\item evol\-\_\-field.\-dat \-: the recorded time evolution of the variables.
\item mean\-\_\-field.\-dat \-: the mean field (the climatology)
\end{DoxyItemize}

The tangent linear and adjoint models of M\-A\-O\-O\-A\-M are provided in the \hyperlink{classmaooam__tl__ad}{maooam\-\_\-tl\-\_\-ad} and \hyperlink{classtl__ad__integrator}{tl\-\_\-ad\-\_\-integrator} module. It is documented \href{./md_README_TL_AD.html}{\tt here}. 



\subsection*{Implementation notes}

As the system of differential equations is at most bilinear in $y_j$ ( $j=1..n$), $\boldsymbol{y}$ being the array of variables, it can be expressed as a tensor contraction \-:

\[ \frac{d y_i}{dt} = \sum_{j,k=0}^{ndim} \, \mathcal{T}_{i,j,k} \, y_k \; y_j \]

with $y_0 = 1$.

The tensor \hyperlink{classaotensor__def_a0dc43bc9294a18f2fe57b67489f1702f}{aotensor\-\_\-def\-::aotensor} is the tensor $\mathcal{T}$ that encodes the differential equations is composed so that\-:


\begin{DoxyItemize}
\item $\mathcal{T}_{i,j,k}$ contains the contribution of $dy_i/dt$ proportional to $ y_j \, y_k$.
\item Furthermore, $y_0$ is always equal to 1, so that $\mathcal{T}_{i,0,0}$ is the constant contribution to $dy_i/dt$
\item $\mathcal{T}_{i,j,0} + \mathcal{T}_{i,0,j}$ is the contribution to $dy_i/dt$ which is linear in $y_j$.
\end{DoxyItemize}

Ideally, the tensor \hyperlink{classaotensor__def_a0dc43bc9294a18f2fe57b67489f1702f}{aotensor\-\_\-def\-::aotensor} is composed as an upper triangular matrix (in the last two coordinates).

The tensor for this model is composed in \hyperlink{classaotensor__def}{aotensor\-\_\-def} and uses the inner products defined in \hyperlink{classinprod__analytic}{inprod\-\_\-analytic} . 



\subsection*{Final Remarks}

The authors would like to thank Kris for help with the lua2fortran project. It has greatly reduced the amount of (error-\/prone) work.

No animals were harmed during the coding process. 
